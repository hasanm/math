\documentclass[12pt]{book}
\usepackage{setspace}
\usepackage{charter}
\usepackage[T1]{fontenc}
\usepackage{url}
% \usepackage{hyperref}
\usepackage{amsmath}
\usepackage{amsfonts}
\usepackage{amssymb}
\usepackage{amsthm}
\usepackage{mathrsfs}
\usepackage{pifont} 
\usepackage[pdftex]{graphicx}
\usepackage{epstopdf}
\usepackage{calc}

\usepackage[hmargin=20mm,top=20mm, bottom=20mm]{geometry}
\usepackage{tikz}
\usetikzlibrary{intersections,arrows,decorations.pathmorphing,backgrounds,positioning,fit,calc,matrix,3d,chains,petri,through,scopes,shadows,shapes.geometric}
\usepackage{tikz-cd}
% \usepackage[normalem]{ulem} % for Strike out. 
\usepackage{soul} % For Strikethrough
\newcommand{\cmark}{\ding{51}}%
\newcommand{\xmark}{\ding{55}}%
\newcommand{\code}{\texttt}
\newcommand\T{\rule{0pt}{2.6ex}}
\newcommand\B{\rule[-1.2ex]{0pt}{0pt}}
\newcommand\Pb{\vspace{1.2ex}}
\newtheorem{note}{Note}[section]
\newtheorem{theorem}{Theorem}
\newtheorem{proposition}{Proposition}
\newtheorem*{problem*}{Problem}
\newtheorem{lemma}{Lemma}
\newcommand{\E}{\mathrm{E}}
\newcommand{\Var}{\mathrm{Var}}
\newcommand\numberthis{\addtocounter{equation}{1}\tag{\theequation}}
\newcommand{\LL}{\mathcal{L}}

% No Indent 
\setlength\parindent{0pt}
\title{Excercise: Linear Algebra\\
\large{Stat 2780, Fall 2024}}
\author{Mahmudul Hasan (hasan@uleth.ca)\\
}
\begin{document}
% \input{ch-intro.tex}
% \maketitle

\chapter{Linear Algebra Done right}

\section{Excercise : 3.D}

\begin{problem*}
  1. Suppose $T \in \LL(U,V)$ and $ S \in \LL(V,W)$ are both invertible
  linear maps. Prove that $ST \in \LL(U,W)$ is invertible and that $(ST)^{-1} = T^{-1}S^{-1}$
\end{problem*}

\begin{proof}
  Since $ST$ is a composition of two bijections, it is also a bijection, and hence is also a bijection. We only need to show that $(ST)^{-1} = T^{-1}S^{-1}$.

  \begin{align*}
    (T^{-1} S^{-1}) (S T) & = T^{-1} (S^{-1} S) T\\
    & = T^{-1} I T\\
    & = T^{-1}T = I \\
  \end{align*}

  Similarly, $(S T) (T^{-1} S^{-1}) = I$.
\end{proof}

\begin{problem*}9. Suppose $V$ is finite-dimensional and $S,T \in \LL(V)$. Prove that $ST$ is invertible if and only if both $S$ and $T$ are invertible.\end{problem*}

\begin{proof}
  The reverse direction is immediate from Problem 1. Now suppose that $ST$ is invertible. Let $v \in V$. Then $STv = v$. Hence $S$ is surjective and therefore invertible. Suppose that $Tu = Tv$. Then, $ST u = ST v$. Since $ST$ is invertible, we have $u = v$. Therefore, $T$ is injective, and since $V$ is finite dimension, $T$ is invertible. 
\end{proof}

\begin{problem*}10. Suppose $V$ is finite-dimensional and $S, T \in \LL(V)$. Prove that $ST = I$ if and only if $TS = I$\end{problem*}

\begin{proof}
  Suppose $ST = I$. Then $STv = v$. Since $V$ is finite dimensional, $S$ is invertible. Now,
  \begin{align*}
    I = S^{-1}S = S^{-1}(ST)S = (S^{-1}S)TS = ITS = TS
  \end{align*}
\end{proof}


\section{Excercise : 7.C}

\begin{problem*}
4. Suppose $T \in \LL(V,W)$ Prove that $T^*T$ is a positive operator on $V$ and $TT^*$ is a positive operator on $W$. 
\end{problem*}

\begin{proof}
  \[
  (T^{*}T)^* = T^*T
  \]
  Therefore, $T^{*}T$ is self-adjoint.

  Moreover,
  \begin{align*}
    \langle T^{*}Tv, v \rangle & = \langle Tv, Tv \rangle\\
    & = ||Tv||^2 \ge 0
  \end{align*}

Therefore, $T^{*}T$ is positive. 
\end{proof}

% ======================================================================
%                       Document Ends here 
% ======================================================================
\end{document}


% Local Variables:
% compile-command: "make -k math4500_init"
% fill-column: 80
% comment-fill-column: 80
% mode: latex
% mode: reftex
% tex-main-file: "webmail.tex"
% End:
